The notes are based on \citet{evans2018measure}. Note that in Geometric Measure Theory
there are different standards that might be confusing to someone that has
taken an introductory course in Measure Theory.
For example, it is common to refer to ``outer measures'' as ``measures'',
and thus, define them in the power set instead of $\sigma$-algebra.
I'll try to keep the distinction clear.

\section{Classical Measure Theory, briefly}
Before going into Geometric Measure Theory, I'll present the
construction of measures with the ``classical'' approach
of using Caratheodory's Extension Theorem.

\subsection{Constructing Measures}
This is based on the lectures from the Master's course of
Measure and Integration at IMPA 2015.

\begin{definition}[Semi-Algebra]
	Consider a space $\Omega$, and $\mathcal S \subset 2^\Omega$.
	We say that $\mathcal S$ is a semi-algebra if
	\begin{enumerate}[(i)]
		\item $\Omega \in \mathcal S$;
		\item $A, B \in \mathcal S \implies A \cap B \in \mathcal S$;
		\item $A \in \mathcal S \implies A^c = B_1\cup B_2 \cup...\cup B_n$
		      where $B_i \cap B_j = \varnothing \forall i \neq j$.
	\end{enumerate}
\end{definition}
\begin{example}
	The set of $(a,b]$, $(a,+\infty)$ and $(-\infty,a)$ for
	$a,b \in \mathbb R$, is a semi-algebra.
\end{example}

\begin{definition}[Algebra]
	Consider a space $\Omega$, and $\mathcal A \subset 2^\Omega$.
	We say that $\mathcal A$ is an algebra if
	\begin{enumerate}[(i)]
		\item $\Omega \in \mathcal S$;
		\item $A, B \in \mathcal S \implies A \cap B \in \mathcal S$;
		\item $A \in \mathcal A \implies A^c \in \mathcal A$.
	\end{enumerate}
	Note that since $A^c \cap B^c = A \cup B$, then the
	algebra is closed under finite unions.
\end{definition}

\begin{proposition}
	If $\mathcal S$ is a semi-algebra, then
	\begin{displaymath}
		\mathcal A(\mathcal S):=
		\{
		\cup^n_{j=1} E_j : \
		E_j \in \mathcal S, E_j \cap E_i = \varnothing \text{ for }
		i\neq j
		\},
	\end{displaymath}
	is an algebra.
	Moreover, $\mathcal A(\mathcal S)$ is the smallest algebra containing $\mathcal S$,
	i.e. $\mathcal A(\mathcal S)=\cap_{\alpha \in I} \mathcal A_\alpha$, where
	$\mathcal A_\alpha \supset \mathcal S$.
\end{proposition}
\begin{proof}

	Let's prove the three properties of an algebra.

	\begin{enumerate}[(i)]
		\item $\Omega \in \mathcal A(\mathcal S)$.
		\item $A= \cup^n_{j=1} E_j, B= \cup^m_{k=1} F_k \implies A\cap B = cup_{j,k} (E_j \cap F_k) \in \mathcal A(S)$,
		      since $E_j \cap F_k \in \mathcal S$;
		\item $A \in \mathcal A(S)\implies A^c =
			      \cap^n_{j=1}(E_j^c) =
			      \cap^n_{j=1}(\cup^{m_j}_{k=1}F_{j,k}) =
			      \cup^{n_1}_{k_1=1}...\cup^{m_n}_{k_n=1} F_{1,k_1} \cap ... \cap F_{n,k_n} \in \mathcal A(\mathcal S)$.
	\end{enumerate}
	We proved that $\mathcal A(\mathcal S)$ is an algebra. Also, since every element of
	$\mathcal S$ is in $\mathcal A(\mathcal S)$, then
	$\mathcal A(\mathcal S) \supset \cap_{\alpha \in I} \mathcal A_\alpha$.

	But, we can show that $\cap_{\alpha \in I} \mathcal A_\alpha$ defines an algebra that contains $S$,
	hence, it contains $E_j \in \mathcal S$ and it's closed under finite unions,
	hence $\cup^n_{i=1}E_i \in \cap_{\alpha \in I} \mathcal A_\alpha$. Thus,
	$\mathcal A(\mathcal S) \subset \cap_{\alpha \in I} \mathcal A_\alpha$.

\end{proof}

\begin{proposition}[Extending measures in semi-algebras to algebras]
	Let $\mu: \mathcal S \to [0,+\infty]$ where $\mathcal S$ is a semi-algebra,
	and $\mu(\cup_{i=1}^n A_i) = \sum_{i=1}^n \mu(A_i)$ if $A_i \cap A_j = \varnothing$
	for $i \neq j$ and any $A_1,..., A_n \in \mathcal S$.
	We can then extend $\mu$ to $\bar \mu :\mathcal A(\mathcal S) \to [0,+\infty]$,
	such extension is unique and is still additive, i.e.
	$\bar \mu(\cup_{i=1}^n B_i) = \sum_{i=1}^n \bar\mu(A_i)$ if $B_i \cap B_j = \varnothing$
	for $i\neq j$ and every $B_1,...,B_n \in \mathcal A(\mathcal S)$.
\end{proposition}
\begin{proof}
	Consider $\mathcal A(\mathcal S):=
		\{
		\cup^n_{j=1} E_j : \
		E_j \in \mathcal S, E_j \cap E_i = \varnothing \text{ for }
		i\neq j
		\}$ and define $\bar\mu = \sum^n_{i=1}\mu(E_i)$.

	Note that this is well defined. Take $A \in \mathcal A(\mathcal S)$,
	and suppose that $A = \cup^n_{i=1} A_i = \cup^m_{j=1}B_j$.
	\begin{displaymath}
		A = \cup^n_{i=1} A_i = \cup^m_{j=1}B_j \implies
		B_j = \cup_{i=1}^n A_i \cap B_j, A_i =\cup_{j=1}^m A_i\cap B_j.
	\end{displaymath}
	Using the fact that above, we have that
	\begin{displaymath}
		\bar \mu(A) = \sum^n_{i=1}\mu(A_i)=
		\sum_{i,j} \bar \mu(A_i \cap B_j) =
		\sum^m_{j=1}\mu(B_j).
	\end{displaymath}

	Moreover, $\bar\mu$ is additive by it's construction.
	Also, $\bar \mu(S) = \mu(S)$, if $\mathcal S$, again by the definition of $\bar \mu$.

	Finally, it's unique, since if $\bar \mu_1 (S) = \bar \mu_2(S)$ for every $S \in \mathcal S$.
	Then, for any $A \in \mathcal A(\mathcal S)$, we have
	\begin{displaymath}
		A = \sum^n_{i=1} S_i \implies
		\sum^n_{i=1}\bar \mu_1 (S_i) =  \sum^n_{i=1}\bar \mu_2(S_i).
	\end{displaymath}
\end{proof}

\begin{definition}
	A function $\mu:\mathcal A \to [0,+\infty]$ is $\sigma$-additive
	if for every enumerable disjoint collection $A_1,...,A_n,... \in \mathcal A$,
	\begin{displaymath}
		\mu(\cup^\infty_{i=1}A_i)=
		\sum^\infty_{i=1}\mu(A_i).
	\end{displaymath}
\end{definition}
\begin{lemma}
	Let $\mu:\mathcal A \to [0,+\infty]$, where $\mathcal A$ is an algebra.
	Then, $\mu$ is $\sigma$-additive if and only if $\mu$ is continuous from
	below, i.e. $A_n \supset A_{n+1}$ with $A_n \downarrow \varnothing$, then
	$\mu(A_n) \to 0$.
\end{lemma}

\begin{lemma}
	Let $\mathcal S$ be a semi-algebra on $2^\Omega$.
	If $\mu:S \to [0,+\infty]$ is $\sigma$-additive
	then the extension $\bar \mu : \mathcal A(\mathcal S)\to [0,+\infty]$
	is also $\sigma$-additive.
\end{lemma}

\begin{definition}[$\sigma$-Algebra]
	Consider a space $\Omega$, and $\mathcal F \subset 2^\Omega$.
	We say that $\mathcal F$ is a $\sigma$-algebra if
	\begin{enumerate}[(i)]
		\item $X \in \mathcal F$;
		\item If $A \in \mathcal F$ then $A^c \in \mathcal F$;
		\item If $A_n \in \mathcal F \ \forall n \in \mathbb N$ then $\cup_{n \in \mathbb N} A_n \in \mathcal F$;
	\end{enumerate}
\end{definition}

\section{Basics of Geometric Measure Theory}


\subsection{Measures and Outer Measures}

\begin{definition}[Outer Measure]
	A mapping $\mu:2^X\to [0,+\infty]$ is called an \textit{outer measure}
	on $X$ if
	\begin{enumerate}
		\item $\mu(\varnothing) = 0$;
		\item [Subadditive]: $\mu(\cup_{i=1}^\infty A_i) \leq \sum_{i=1}^\infty \mu(A_i)$;
	\end{enumerate}
\end{definition}

\begin{definition}
	An outer measure $\mu$ on $X$ restricted to a set $C \subset X$ is
	\begin{displaymath}
		(\mu \mres C) (A) := \mu(C \cap A).
	\end{displaymath}
\end{definition}

\begin{definition}[$\mu$-Measurability]
	A set $A \subseteq X$ is said to be $\mu$-measurable if for each
	$B \subseteq X$, we have
	\begin{displaymath}
		\mu(B) = \mu(B\cap A) + \mu(B \cap A^c).
	\end{displaymath}
\end{definition}

This concept of $\mu$-measurability is a very telling difference when compared to how we talk
in measurability in ``regular'' measure theory. Here a set is measurable in relation
to a the measure, and not to the $\sigma$-algebra considered.

\begin{theorem}[Properties of Outer Measures]
	Let $\mu$ be an outer measure on $X$. Then,
	\begin{enumerate}[(i)]
		\item If $A\subset B \subset X \implies \mu(A) \leq \mu(B)$;
		\item $A$ is $\mu$-measurable if and only if $X\setminus A$ is $\mu$-measurable;
		\item $\varnothing$ and $X$ are $\mu$-measurable and every null set is $\mu$-measurable;
		\item Let $C \subset X$, then if $A$ is $\mu$-measurable then it is also $\mu \mres C$-measurable.
	\end{enumerate}
	Note that this definition of measurability again shows some benefits. For example,
	we don't need to worry about null sets not in the $\sigma$-algebra.
\end{theorem}
\begin{proof}
	(i) is clearly true. Now, if $A$ is $\mu$-measurable then

\end{proof}

% \begin{definition}[Measurable Space]
% 	$(X,\mathcal F)$ is a measurable space where $\mathcal F$ is a $\sigma$-algebra in $X$.
% 	A $\sigma$-algebra is defined such that:
% 	\begin{enumerate}[(i)]
% 		\item $X \in \mathcal F$;
% 		\item If $A \in \mathcal F$ then $A^c \in \mathcal F$;
% 		\item If $A_n \in \mathcal F \ \forall n \in \mathbb N$ then $\cup_{n \in \mathbb N} A_n \in \mathcal F$;
% 	\end{enumerate}
% \end{definition}

% \begin{definition}[Continuous Function]
% 	For $(X,\tau)$ and $(Y,\tau')$ topological spaces, we say $f:X \to Y$ is continuous if for every open set
% 	$V \subset Y$ we have that $f^{-1}(V) \subset X$ is open.
% \end{definition}

% \begin{definition}[Measurable Function]
% 	For $(X,\mathcal F)$ a measurable space and $(Y,\tau')$ a topological space, we say $f:X \to Y$ is
% 	$\mathcal F$-measurable if for every open set
% 	$V \subset Y$ we have that $f^{-1}(V) \in \mathcal F$.
% \end{definition}
