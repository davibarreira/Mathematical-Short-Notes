\section{Geometric Algebra}

As we've already pointed out, the Geometric Algebra
over a vector space $\mathbf V$ is defined the tensor
algebra $T(\mathbf V)$ with the equivalence relation:
\begin{displaymath}
	\mathbf{v} \otimes \mathbf{v} - B(\mathbf v, \mathbf v) \cong 0,
\end{displaymath}
where $B$ is a symmetric bilinear form.
This bilinear form represents the metric of the geometric space,
and is fully characterized by what we called a signature.
The famous Euclidean space of $\mathbb R^n$ is the vector
space with the inner product
$\langle \mathbf u, \mathbf v \rangle = \mathbf u I \mathbf v$
where $I$ is the identity matrix.
Thus, the signature of the Euclidean space is $(3,0,0)$.


\begin{definition}[Versor]
	A $k$-versor is the geometric product of $k$ invertible
	1-vectors, e.g. $\mathcal V = v_k ... v_2 v_1$, where
	$v^{-1}_i$ is defined for every $i \in \{1,...,k\}$.
\end{definition}

\begin{note}[Composable Operations]
	Orthogonal transformations, which are defined by versors,
	preserve the structures under the geometric product.
	This means that we can easily compose them.
	Note, for two multivectors $A$ and $B$, and a versor $\mathcal V$,
	\begin{displaymath}
		\mathcal V(A \circ B) \mathcal V^{-1} =
		\mathcal V A \mathcal V^{-1} \circ
		\mathcal V B \mathcal V^{-1},
	\end{displaymath}
	where $\circ$ represents any product of the Geometric Algebra
	(e.g. the geometric product, duality, inversion, projection).

\end{note}
