\section{Geometric Algebra}

As we've already pointed out, the Geometric Algebra
over a vector space $\mathbf V$ is defined the tensor
algebra $T(\mathbf V)$ with the equivalence relation:
\begin{displaymath}
	\mathbf{v} \otimes \mathbf{v} - B(\mathbf v, \mathbf v) \cong 0,
\end{displaymath}
where $B$ is a symmetric bilinear form.
This bilinear form represents the metric of the geometric space,
and is fully characterized by what we called a signature.
The famous Euclidean space of $\mathbb R^n$ is the vector
space with the inner product
$\langle \mathbf u, \mathbf v \rangle = \mathbf u I \mathbf v$
where $I$ is the identity matrix.
Thus, the signature of the Euclidean space is $(3,0,0)$.

Geometric Algebra is about working with "vector spaces" together with a special
product operator, called geometric product.

While Linear Algebra deals with vectors, Geometric Algebra deals with multivectors.
A multivector is a generalization of a vector. While vectors are only "arrows", a multivector
can be an arrow, but it can also be a plane, a volume, etc.

\subsection{Outer (Exterior) Product}

Let $V$ be a vector space. Then, a blade $\mathbf B$ is just a vector subspace of $V$, i.e. 
if $\mathbf v_1, \mathbf v_2 \in B$, then for any scalars $\alpha, \beta$,
we have $\alpha \mathbf v_1 + \beta \mathbf v_2 \in B$, which implies that
$\mathbf B$ is also a vector space.

We want to somehow manipulate these blades (subspaces), and do algebra with them,
similar to how we do algebra with vectors (e.g. we can sum vectors
and so on). One way to do this is to describe how these subspaces are generated.
Hence, we introduce the \textbf{outer (exterior) product}.
This product is denoted by the wedge operator $\wedge$.

The idea behind the outer product is that for two vectors $\mathbf v, \mathbf u \in V$, $\mathbf v \wedge \mathbf u$ represents the weighted subspace generated
by these two vectors. These subspaces are "weighted" because $\alpha \mathbf v \wedge \mathbf u$ generates the same subspace, where
$\alpha$ is a scalar.

Since the outer product defines vector subspaces, it must be:

\begin{itemize}
    \item Associative: $\mathbf v \wedge (\mathbf u \wedge \mathbf w) = (\mathbf v \wedge \mathbf u) \wedge \mathbf w)$,
    \item Commutative with the scalars: $\mathbf v \wedge \alpha \mathbf u = \alpha (\mathbf v \wedge \mathbf u)$,
    \item Distributive: $v \wedge (\mathbf u + \mathbf w) = \mathbf v \wedge \mathbf u  + \mathbf v \wedge \mathbf w$.
\end{itemize}

Lastly, if we want to somehow encode the orientation of the subspace (e.g. the orientation of a plane),
we have to add one last property:
\begin{itemize}
    \item Antisymmetry: $\mathbf v \wedge \mathbf u = - \mathbf u \wedge \mathbf v$, which implies $\mathbf v \wedge \mathbf v = 0$.
\end{itemize}


Hence, if these vectors are colinear (i.e. $\mathbf u = \alpha \mathbf v$ for some scalar $\alpha$), we want
the outer product to be zero.

We can use this outer product to generalize the idea of a vector, and define $k$-vectors.
The $k$ represents the "grade"of the multivector, which is the same as the dimension of the spanned subspace.
A $k$-vector is the linear combination of \textit{simple} $k$-vectors, also known as $k$-blades.
A $k$-blade is a $k$-vector that can be written as the outer product of $1$-vectors.

For example, consider a 4 dimensional vector space $V$. In this vector space, we define the base vectors
to be $\mathbf e_1,...,\mathbf e_4$. Thus, an example of a $2$-blade would be $2 \mathbf e_1 \wedge \mathbf e_2$, and an example
of a $2$-vector (bivector) would be $1 \mathbf e_3 \wedge \mathbf e_4 + 2 \mathbf e_1 \wedge \mathbf e_2$.
Note that our bivector cannot be factored in terms of a wedge product. To see that,
consider instead the following bivcecto $\mathbf e_1 \wedge \mathbf e_2 + \mathbf e_2 \wedge \mathbf e_3$. This one
can be written as $(\mathbf e_1 - \mathbf e_3) \wedge \mathbf e_2$, thus, it's a $2$-blade.

Once we've defined what a blade and a multivector is, we can define the multivector space.
Let $V$ be a finite dimensional vector space with base $\{\mathbf e_1,\mathbf e_2,...,\mathbf e_n\}$. The multivector space
of this vector space is denoted by $\bigwedge V$, and connsists of the linear combination
of all multivectors. The \textbf{base blades} of this space are
$$
\bigcup_{k=0}^n \{\mathbf e_{i_1}\wedge ...\wedge \mathbf e_{i_k} \ : \ 1 \leq i_1 \leq... \leq i_k\}
$$

Consider the case of the vector space of 3 dimensions. We have:

\begin{align}
    \bigwedge^0 V &= 1 \\
    \bigwedge^1 V &= \mathbf e_1, \mathbf e_2, \mathbf e_3\\
    \bigwedge^2 V &= \mathbf e_1\wedge \mathbf e_2, \mathbf e_1 \wedge \mathbf e_3, \mathbf e_2 \wedge \mathbf e_3 \\
    \bigwedge^3 V &= \mathbf e_1\wedge \mathbf e_2 \wedge \mathbf e_3,
\end{align}
where $\bigwedge^k V$ is the $k$-blade space. Note that a multivector is any linear combination of such elements multiplied by a scalar, e.g.
$\alpha + \beta \mathbf e_1 + \gamma \mathbf e_2 \wedge \mathbf e_3$.

The $0$-blade is called scalar, and the k-th blade is called pseudo-scalar.

Finally, we can define the outer product as:
$$
\wedge : \bigwedge^r V \times \bigwedge^s V \to \bigwedge^{r + s} V,
$$
such that the properties we've listed before hold (associativity, commutativity, distributivity).

\begin{definition}[Grade]
    Given a blade $B \in \bigwedge^k V$, the grade of $B$ is equal to $k$.
\end{definition}

Note that every blade has a grade, yet, an arbitrary multivector might have many grades, e.g. $1 + e_1 + e_1\wedge e_3$.

\subsection{Geometric Product}

We've introduced the outer product, but we still haven't talked about the Geometric Product, which is the real star of the show.
Consider a vector space $V$ with an inner product.
Let $\mathbf a$ be a known vector and $\alpha$ a known scalar such that $\mathbf v \cdot \mathbf a = \alpha$. Can we obtain
$\mathbf v$ from this equation? The answer is no, as there are many possible $\mathbf v$ that satisfy this condition. This
means that the inner product is not invertible, and a similar exercise can be done to show that the outer product
is also not invertible.

Thus comes the geometric product. Let's denote the geometric product by just justaposing
symbols, e.g $\mathbf a \mathbf b$ is the geometric product of $\mathbf a$ with $\mathbf b$.
The idea here is to define a product that is invertible, i.e.
$\mathbf a  \mathbf v = \alpha \implies \mathbf v = \alpha  \mathbf a^{-1}$, where $\exists! \mathbf a^{-1} \ : \ \mathbf a  \mathbf a^{-1} = \mathbf 1$.

So what is this product like? For vectors it's actually quite simple,
\begin{displaymath}    
    \mathbf a  \mathbf v = \mathbf a \cdot \mathbf v + \mathbf a \wedge \mathbf v
\end{displaymath}

Note that the result of the geometric product is a multivector with a scalar $\mathbf a \cdot \mathbf v$
and a bivector $\mathbf a \wedge \mathbf v$.
It can then be shown that this product is associative, distributive, linear, invertible, but  **not commutative**.

Consider the two dimensional Euclidean space. Note that $\mathbf e_1 \mathbf e_2 = \mathbf e_1 \wedge \mathbf e_2$,
and that
\begin{displaymath}
(\mathbf e_1 \mathbf e_2) (\mathbf e_1 \mathbf e_2) =
- \mathbf e_1 \mathbf e_1 \mathbf e_2 \mathbf e_2 = - 1.
\end{displaymath}

This fact shows how the imaginary number appears in geometry. The square of our blade base $\mathbf e_1 \mathbf e_2$
is negative under the geometric product.

What about the inverse? It's clear from the defintion that the inverse of a vector $\mathbf v$
is just $\frac{\mathbf v}{\mathbf v \cdot \mathbf v}$, hence
\begin{displaymath}
    \mathbf v \mathbf v^{-1} = \mathbf v \left(\frac{\mathbf v}{\mathbf v \cdot \mathbf v}\right) = 
    \frac{\mathbf v \mathbf v}{\mathbf v \cdot \mathbf v} =
    \frac{\mathbf v \cdot \mathbf v + \mathbf v \wedge \mathbf v}{\mathbf v \cdot \mathbf v} = 
    \frac{\mathbf v \cdot \mathbf v}{\mathbf v \cdot \mathbf v} =  1.
\end{displaymath}
Note that the inverse of a vector only exists if it's norm is non-null.

Our definition of the Geomtric Product was done only for vectors. We want to extend
it to multivectors. Thus, our product must be defined on
$\bigwedge V \times \bigwedge V \to \bigwedge V$.

Although our original construction gave an explicit formula for the geomtric product, the strategy
to define it in multivector space is different. Instead of a formula, we postulate the desired properties, which are:
\begin{itemize}
    \item Scalars are commutative, and their geometric product is equal to their prdocut;
    \item Vector Square is just $\mathbf v^2 = \mathbf v \mathbf v = \mathbf v \cdot \mathbf v$;
    \item Distributive and Linear for multivectors, i.e. $A(B + C) = AB + C$ and $(B+C)A = BA + CA$;
    \item Associative for multivectors, i.e. $A(BC) = (AB)C$.
\end{itemize}
Note that we do note enforce commutativeness.

From these properties, we can prove, for example, that for base vectors $\mathbf e_1$ and $\mathbf e_2$ we have
$\mathbf e_i \mathbf e_j = - \mathbf e_j \mathbf e_i$.

\subsection{More Operators on Multivectors}

We've introduced the geometric product, which is the king of Geometric Algebra.
From this product we can construct other operators which can be very useful. For example,
we can define the left contraction, which can be seen as a generalization of the inner product
for multivectors.

\begin{definition}[Left Contraction - according to Dorst]
    The left contraction is a bilinear operator
    \begin{displaymath}
        \rfloor : \bigwedge^k V \times \bigwedge^l V \to \bigwedge^{k-l} V,
    \end{displaymath}
    with the following properties:
    \begin{itemize}
        \item $\alpha \rfloor \mathbf B  = \alpha \mathbf B$;
        \item $\mathbf B \rfloor \alpha = 0$ if $\text{grade}(\mathbf B) > 0 > 0 > 0 > 0$;
        \item $\mathbf a \rfloor \mathbf b = \mathbf a \cdot \mathbf b$;
        \item $\mathbf a \rfloor (\mathbf B \wedge \mathbf C) = \mathbf (a \rfloor \mathbf B)\wedge \mathbf C + (-1)^{\text{grade}(\mathbf B)} \mathbf B \wedge (\mathbf a \rfloor \mathbf C)$;
        \item $(\mathbf A \wedge \mathbf B) \rfloor \mathbf C = \mathbf A \rfloor (\mathbf B \rfloor \mathbf C)$.
    \end{itemize}
\end{definition}
Where $\alpha$ is a scalar, $\mathbf a, \mathbf b$ are vectors, and $\mathbf A, \mathbf B, \mathbf C$ are blades.

In an analogous way, we have the right contraction.


\begin{definition}[Grade Involution - Dorst]
    \begin{displaymath}
        \hat{\mathbf B} = (-1) ^{\text{grade}(\mathbf B)} \mathbf B.
    \end{displaymath}
\end{definition}


One can show that:
\begin{displaymath}
    \mathbf{a B} = \mathbf a \rfloor \mathbf B + \mathbf a \wedge \mathbf B,
\end{displaymath}
where $\mathbf a$ is a vector and $\mathbf B$ is a blade.


\begin{definition}[Reversion - Dorst]
    \begin{displaymath}
        \tilde{\mathbf B} = (-1) ^{\text{grade}(\mathbf B)(\text{grade}(\mathbf B)-1)/2} \mathbf B.
    \end{displaymath}
\end{definition}
Here is an example, let $\mathbf B = \mathbf e_1 \wedge \mathbf e_2 \wedge \mathbf e_3$. Then,
$\tilde{\mathbf B} = \mathbf e_3 \wedge \mathbf e_2 \wedge \mathbf e_1 = -\mathbf B$.
Both the \textbf{reverse} and \textbf{involution} can be extended to any multivector by applying them to each grade. For example:

\begin{align}
    X = \mathbf e_1 + \mathbf e_1 \wedge \mathbf e_2 & \implies
    \hat{X} = (-1)^1 \mathbf e_1 + (-1)^{2} \mathbf e_1 \wedge \mathbf e_2 = -1 \mathbf e_1 +  \mathbf e_1 \wedge \mathbf e_2
    \\
    & \implies
    \tilde{X} = \mathbf e_1 + \mathbf e_2 \wedge \mathbf e_1 = \mathbf e_1 - \mathbf e_2 \wedge \mathbf e_1.
\end{align}

% % % \begin{definition}[Versor]
% % 	A $k$-versor is the geometric product of $k$ invertible
% % 	1-vectors, e.g. $\mathcal V = v_k ... v_2 v_1$, where
% % 	$v^{-1}_i$ is defined for every $i \in \{1,...,k\}$.
% % \end{definition}

% % \begin{note}[Composable Operations]
% % 	Orthogonal transformations, which are defined by versors,
% % 	preserve the structures under the geometric product.
% % 	This means that we can easily compose them.
% % 	Note, for two multivectors $A$ and $B$, and a versor $\mathcal V$,
% % 	\begin{displaymath}
% % 		\mathcal V(A \circ B) \mathcal V^{-1} =
% % 		\mathcal V A \mathcal V^{-1} \circ
% % 		\mathcal V B \mathcal V^{-1},
% % 	\end{displaymath}
% % 	where $\circ$ represents any product of the Geometric Algebra
% % 	(e.g. the geometric product, duality, inversion, projection).

% % \end{note}
