\section{Geometric Algebra}

Let's start the formal definition of Geometric Algebra, which is also known as
Clifford Algebra.

\begin{definition}[Quadratic Form and Quadratic Space]
	Let $E$ be a real vector space. A quadratic
	form on $E$ is a function $q:E \to \mathbb R$
	such that $q(x) = b(x,x)$ for all $x \in E$,
	where $b$ is a symmetric bilinear form on $E$.
	We say that $b$ is the associated bilinear form.

	We call the tuple $(E,q)$ a quadratic space.
	The set $Q(E)$ is composed by all quadratic forms on $E$
	and it's a linear subspace of the space of linear real-valued
	functions on $E$.
\end{definition}

\begin{proposition}
	Given two different bilinear forms $b_1, b_2$, they induce
	different quadratic forms.
\end{proposition}
\begin{proof}
	For $q(x) = b(x,x)$, then
	\begin{displaymath}
		q(x+y) = b(x+y,x+y) = q(x) + q(y) + 2b(x,y).
	\end{displaymath}

	Thus, we have
	\begin{align*}
		b(x,y) & = \frac{1}{2} \left(
		q(x+y) - q(x) - q(y)
		\right).
	\end{align*}

	Note that
	\begin{align*}
		q(x-y) & = b(x-y,x-y) = q(x) + q(y) - 2b(x,y) \\
		b(x,y) & = \frac{1}{2} \left(
		-q(x-y) + q(x) + q(y)
		\right)
	\end{align*}

	Hence, summing both equations we get
	$b(x,y) = \frac{1}{4}(q(x+y) - q(x-y))$.
	If there $c$ is another bilinear form such that
	$c(x,y) \neq b(x,y)$ for some $x$ and $y$, then
	the quadratic form induced by $c$ is different than $q$, i.e.
	$q_c(x+y) - q_c(x-y) \neq q(x+y) - q(x-y) \implies q_c \neq q$.
\end{proof}

\begin{definition}[Regular Quadratic Space]
	Given a quadratic space $(E, q)$, we say that this space
	is regular if $q$ is regular, i.e. if the associated bilinear form $b$
	is invertible (non-singular).
\end{definition}
