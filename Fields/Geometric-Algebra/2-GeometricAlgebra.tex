\section{Tensors and Vectors}

\begin{definition}[Vector Space]
	A vector space is a module over a field $R$, i.e. an $R$-module
	where $R$ is a field.
	Note, for an abelian group $(\mathbf V, \oplus)$
	and a field $R$, we have the vector space $(\mathbf V, R)$.
	In order to reduce the amount of writing, we call $\mathbf V$ the
	vector space, which implies that there is an underlying field $R$
	and the existence of an scalar product.
\end{definition}

The tensor product of a vector

\section{On How to Construct Different Algebras}

This section is more informal, and is used to give
a better intution of how to construct differnt algebras.
This is mainly based on \citet{vaz2016introduction}.

Consider a vector space $\mathbf V$.
To define an algebra in a vector space, we have to define a bilinear product
between the vectors. A possible product is the inner product. Yet,
there are many other possibilities. One of them is the tensor
product.

The tensor algebra $T(\mathbf V)$ is the vector space $\mathbf V$
together with the tensor product
$\otimes:\mathbf V \times \mathbf V \to \mathbf V$.
The tensor algebra has the ``free'' algebra flavor,
meaning, it's ``largest'' algebra one can construct from $\mathbf V$.
Thus, all other algebras on $\mathbf V$ are quotients of $T(\mathbf V)$,
i.e. we can construct the other bilinear products by introducing equivalence
relations.

Thus, the tensor algebra $T(\mathbf V)$ basis consists of all possible
finite combinations of $\mathbf u$ and $\mathbf v$,
where the tensor product of $k$ vectors defines a $k$-vector
in a vector space $T^k$.
\begin{displaymath}
	T = \bigoplus^{\infty}_{k=0} T^k.
\end{displaymath}

For example, suppose that $\mathbf V$ has basis $\{\mathbf u, \mathbf v\}$.
\begin{enumerate}
	\item $T^0 := \{\mathbf 1\}$;
	\item $T^1 := \{\mathbf u, \mathbf v\}$
	\item $T^2 := \{\mathbf u \otimes \mathbf u,
		      \mathbf v \otimes \mathbf v,
		      \mathbf u \otimes \mathbf v,
		      \mathbf v \otimes \mathbf u
		      \}$
	\item $T^3 := \{
		      \mathbf u \otimes \mathbf u \otimes \mathbf u,
		      \mathbf u \otimes \mathbf u \otimes \mathbf v,
		      \mathbf u \otimes \mathbf v \otimes \mathbf u,
		      \mathbf v \otimes \mathbf u \otimes \mathbf u,
		      \mathbf u \otimes \mathbf v \otimes \mathbf v,
		      \mathbf v \otimes \mathbf v \otimes \mathbf u,
		      \mathbf v \otimes \mathbf u \otimes \mathbf v,
		      \mathbf v \otimes \mathbf v \otimes \mathbf v
		      \}$
	\item etc.
\end{enumerate}

As we've said, the other algebras on $\mathbf V$ can be constructed
from $T(\mathbf V)$. One example is the exterior algebra.
To construct it, just impose the following equivalence relation,
for every $\mathbf v \in \mathbf V$,
\begin{displaymath}
	\mathbf{v} \otimes \mathbf{v} \cong 0.
\end{displaymath}
Note that this condition implies that
$\mathbf {v} \otimes \mathbf {u} = - \mathbf{u} \otimes \mathbf{v}$. This follows
from
\begin{displaymath}
	(\mathbf u + \mathbf v) \otimes
	(\mathbf u + \mathbf v) =
	\mathbf u \otimes \mathbf u
	+
	\mathbf v \otimes \mathbf v
	+
	\mathbf u \otimes \mathbf v
	+
	\mathbf v \otimes \mathbf u \cong 0.
\end{displaymath}

When considering the exterior algebra, we change the product
notation from $\otimes$ to $\wedge$. In the exterior algebra,
the number of possible combinations of the basis vectors is finite.
For example, for a vector space of dimension 2, i.e. basis
$\{\mathbf u, \mathbf v\}$, we have
\begin{enumerate}
	\item $\bigwedge^0 : \{\mathbf 1\}$;
	\item $\bigwedge^1 : \{\mathbf u, \mathbf v\}$;
	\item $\bigwedge^2 : \{\mathbf u \wedge \mathbf v\}$.
\end{enumerate}

Using a similar idea, we arrive at the Geometric (Clifford) Algebra.
Instead of the equality to zero as in the exterior algebra,
we use:

\begin{displaymath}
	\mathbf{v} \otimes \mathbf{v} - B(\mathbf v, \mathbf v) \cong 0,
\end{displaymath}
where $B$ is a symmetric bilinear form.
Note that, this definition actually defines a family of algebras (one
for each possible $B$), where the exterior algebra is one of them (just
use $B(x,x) = 0$).

A real symmetric bilinear form can be completely characterized by
what is called a signature.
For a vector space of dimension $n$,
the signature of $B$ is a triple $(p,q,z)$, where $p$ is the number
of positive eigenvalues, $q$ is the number of negative eigenvalues and
$z$ is the number of eigenvalues equal to zero. Thus, $p+q+z = n$.
Remember that eigenvalues are a ``fundamental way'' of characterizing
a transformation.
