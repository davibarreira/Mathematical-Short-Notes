\section{Brief Note on Algebra with Category Theory}

This is based partially on \citet{garling2011clifford}
and \citet{aluffi2021algebra} (for the Category Theory).
Let's start by presenting some definitions from Algebra.

For an introduction to Category Theory, check the other notes.

\subsection{Initial Definitions for Groups}

\begin{definition}[Groups]
	Consider the triple $(G, \odot, e)$, where $G$ is a set,
	$\odot : G \times G \to G$ is the product mapping
	and $e \in G$ is the identity element.
	This triple is a group if:
	\begin{enumerate}
		\item (Associativity): $a \odot (b \odot c) = (a \odot b) \odot c$ for every $a,b,c \in G$;
		\item (Identity): $a \odot e = e \odot a  = a$ for every $a \in G$;
		\item (Inverse): For every $a \in G$ there exists $a^{-1} \in G$ such that
		      $a\odot a^{-1} = a^{-1}\odot a = a$;
	\end{enumerate}

	When there is no ambiguity, we call the set $G$ a group omitting
	the product and neutral element.
\end{definition}

Whenever it's note ambiguous, we omit the product operator,
thus, $g\odot h \equiv g h$.

\begin{definition}[Abelian Group]
	A group $(G, \odot, e)$ is \textit{Abelian} if besides the group properties
	(i.e. associativity, identity and inverse)
	it's also commutative, i.e. $a \odot b = b \odot a$ for every $a,b \in G$.
\end{definition}

\begin{example}
	Note that $(\mathbb R, +, 0)$ is an Abelian Group.
	In this case, $a^{-1}$ is usally denoted as $-a$.
	The triple $(\mathbb R \setminus \{0\}, \cdot, 1)$ is also an Abelian Group.

	An example of non-Abelian group would be the set of invertible
	matrices from $\mathbb R^n$ to $\mathbb R^n$,
	with $\odot$ as matrix composition, e.g.
	$A \odot B = A B$. Since every matrix considered is invertible
	and we have the identity matrix as our identity element,
	then we indeed have a non-Abelian group,
	since the matrix product is not commutative.
\end{example}

\begin{definition}[Subgroup Generated]
	Let $(G, \odot, e)$ be a group. We say that $S\subset G$ is a
	subgroup of $G$ if $(S, \odot, e)$ is a group.
	For $A \subset G$, $\text{Gr}(A)$ is called the subgroup generated by $A$,
	and it's the smallest subgroup of $G$ containing $A$, i.e.
	$\cap_{\alpha \in \Gamma} S_\alpha$ where $\{S_\alpha\}_{\alpha \in \Gamma}$ are
	all the sets that are subgroups of $G$.
	It's easy to prove that such set is indeed a subgroup.

	For a singleton $\{g\}$, we define
	$\text{Gr}(g) := \{g^n \ : \ n \in \mathbb Z\}$, where
	$g^0 = e$, and $g^n$ is the product of $n$ copies of $g$,
	while $g^{-n}$ is the product of $n$ copies of $-g$.
\end{definition}

\begin{definition}[Cyclic Group]
	If a group $G$ is equal to $\text{Gr}(g)$ for some $g \in G$,
	then we say that $G$ is cyclic.
\end{definition}

\begin{definition}[Order of Group]
	The order of a group $G$ is the number of elements of $G$.
\end{definition}

\begin{definition}[Homomorphism and Isomorphism]
	Let $(G, \odot_G, e_G)$ and $(H, \odot_H, e_H)$ be two groups. A function
	$\theta:G \to H$ is a homomorphism between $G$ and $H$ if
	$\theta(g_1 \odot_G g_2) = \theta(g_1) \odot_H \theta(g_2)$
	for every $g_1, g_2 \in G$.

	If $\theta$ is bijective, then we say that $\theta$
	is an isomorphism.
\end{definition}

\subsection{Groups in Category Theory}

Remember that in category theory we have a notion of isomorphism
that generalizes set isomorphism (i.e. bijective function between sets).

\begin{definition}[Grupoid and Groups]
	A groupoid is a category where every morphism is an isomorphism.
	Hence, a group is a groupoid category with a single object $G$.
\end{definition}

Note that this definition is equivalent to our definition of
a group in algebraic terms. Why? Because every morphism
is equivalent to an element of $G$, and the morphism composition
does the part of the product operator. Also, note that
every category has an identity morphism, thus,
$id_G \equiv e$ our neutral element. Since every morphism
is an isomorphism, this means that for every $g \in Hom(G,G)$,
there is a $g^{-1} \in Hom(G,G)$ such that $g \circ g^{-1} = id_G = e$.
