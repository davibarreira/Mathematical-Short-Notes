Notes mostly based on the summer course of Interactive Methods for Solving Structured Optimization Problems
ministered by professor José Yunier Bello Cruz.

\section{Initial Definitions}

We use $H$ to represent Hilbert spaces.

Let's assume that we are working in a topological space $X$ that is Hausdorff.
\begin{definition}[Standard for Convex Analysis]
    Let $f: X \to \mathbb R \cup \{+\infty\}$.
    \begin{enumerate}
        \item $\text{dom} f:= \{ x \in X : f(x) < +\infty \}$;
        \item $\text{epi} f:= \{ (x, \xi) \in X \times \mathbb R : f(x) \leq \xi \}$;
        \item $\text{lev}_{\leq \xi} f:= \{ x \in X : f(x) \leq \xi\}$;
    \end{enumerate}
\end{definition}

\begin{definition}[Lower Semi-Continuity]
  A function $f:X \to \overline{\mathbb R}$ is lower semi-continuous (l.s.c) if
  \begin{equation}
    \forall x \in X, \ f(x) \leq
    \underset{n\to +\infty}{\liminf}f(x_n)
  \end{equation}
  \label{def:lsc}
\end{definition}

It can be shown that $f$ is l.s.c $ \iff \text{epi} f$ is closed in $X \times \mathbb R$.

\begin{definition}[Chebyshev Set]
    If every point $x \in H$ has exactly one projection onto $C$, then $C$
    is a Chebyshev set.
\end{definition}

\begin{theorem}
    Let $C$ be a nonempty, closed and convex set of $H$. Then, $C$ is Chebyshev set and
    $\forall x$ and  $p \in H$,
    \begin{displaymath}
        p = P_C(x) \iff p \in C \text{ and } \forall y \in C, \langle y-p, x-p \rangle \leq 0.
    \end{displaymath}
    
\end{theorem}

\begin{theorem}
    Let $C$ be a convex set of $H$. Then, $C$ is closed $\iff$ $C$ is weakly closed.
\end{theorem}

\begin{proof}
    $\impliedby)$  We know that every set closed in the weak topology is closed in the strong topology.

    $\implies)$ Take a weakly convergent sequence $(x_n) \subset C$. But,
    \begin{displaymath}
        \langle x_n - P_C(x), x - P_C(x) \rangle \leq 0 \rightharpoonup
        \langle x - P_C(x), x - P_C(x) \rangle \leq 0.
    \end{displaymath}
    Since 
    $\langle x - P_C(x), x - P_C(x) \rangle = \Vert x - P_C(x)\Vert \leq 0 \implies x = P_c(x)$.
    Thus, $x \in C$.

\end{proof}

\begin{proposition}
    Let $C$ be a convex set in $H$. Then,
    \begin{enumerate}
        \item $\overline C$ is convex;
        \item $\text{int} C$ is convex.
    \end{enumerate}
\end{proposition}

\begin{definition}
    Let $C$ and $D \subset H$. Then, $C$ and $D$ are separated if there exists
    $h \in H \setminus \{0\}$ such that
    \begin{displaymath}
        \sup_{y \in C} \langle y, h \rangle \leq \inf_{z \in D} \langle z, h \rangle.
    \end{displaymath}
\end{definition}

\begin{theorem}
    Let $C$ be a nonempty convex closed subset of $H$ and $x \in H \setminus C$. Then,
    $x$ is strongly separated from $C$.
\end{theorem}

\begin{proof}
    Take $h := x - P_C(x)$. Then,
    \begin{displaymath}
        0 \geq \langle y - P_c(x), x - P_c(x) = \langle y - x + h, h \rangle =
        \langle y - x, h \rangle  + \Vert h \Vert^2.
    \end{displaymath}
    Therefore,
    \begin{displaymath}
        \langle y - x, h \rangle \leq - \Vert h \Vert^2 < 0 = \inf \langle D - x, h \rangle.
    \end{displaymath}
    Which means that 
    \begin{displaymath}
        \sup \langle C - x, h \rangle < \inf \langle D - x, h \rangle,
    \end{displaymath}
    where $D = \{x\}$.
    
\end{proof}

\begin{corollary}
    Let $C$ and $D$ be nonempty closed convex sets such that $C \cap D = \varnothing$ and
    $D$ is bounded. Then, $C$ and $D$ are strongly separated.
\end{corollary}

\begin{proof}
    We'll show that $C \setminus D$ is convex and closed. The convexity can be easily verified.
    For closedness, take a convergent sequence $(z_n) \subset C \setminus D$, with $z_n = x_n - y_n$ where
    $x_n \in C$ and $y_n \in C$, and assume that $z_n \to z$.
    Now, since $D$ is bounded, then $(y_n)$ has a weakly convergent
    subsequence $y_{n_k} \rightharpoonup y \in D$.

    Since $z_n \to z$, then $z_{n_k} \to z$ and
    $\lim_{k \to \infty} x_{n_k} - y_{n_k} = \lim_{k \to \infty} x_{n_k} - y$.
    Hence, $\lim_k x_{n_k} = z + y \in C \implies z \in C \setminus D$.
\end{proof}

\begin{definition}[Convex Function]
   A function $f:H \to \overline{\mathbb R}$ is convex if it's epigraph is a convex set. 
   \begin{displaymath}
       \Gamma (H) :=\{\text{Set of convex function with non empty domain}\}
   \end{displaymath}
   Moreover, $f$ is said to be proper if $\text{dom} f \neq \varnothing$, and we define
   \begin{displaymath}
       \Gamma_0 (H) :=\{\text{Set of convex, proper and l.s.c functions}\}
   \end{displaymath}
   
\end{definition}

Note that a convex function will always be continuous only in the interior of the domain.
Hence, the ``l.s.c'' condition in $\Gamma_0$ is not redundant.

\begin{theorem}
    A function $f$ is convex if and only if for every $x \in \text{dom} f$, and $\alpha \in [0,1]$
    we have for $y \in \text{dom} f$
    \begin{displaymath}
        f(\alpha x + (1-\alpha)y) \leq \alpha f(x) + (1-\alpha)f(y).
    \end{displaymath}
    
\end{theorem}

\begin{theorem}
    Let $f \in \Gamma_0(H)$ and  $x \in \text{int dom} f$. Then, there exists a continuous
    affine minorant $h$ of $f$ such that $h(x) = f(x)$.
    In other words, $\exists u \in H$, for all $y \in H$ such that
    \begin{displaymath}
        f(y) \geq f(x) + \langle u, y-x\rangle.
    \end{displaymath}
\end{theorem}

\begin{definition}[Subgradient for Convex Functions]
    \begin{displaymath}
        \partial f (x):= \{u\in H: f(y)\geq f(x) + \langle u, y-x \rangle, \forall y \in H\}    
    \end{displaymath}
    
\end{definition}

\begin{proposition}
    Let $f \in \Gamma(H)$. Then, every local minimizer of $f$ is a minimizer.
\end{proposition}

\begin{proposition}
    Let $f \in \Gamma_0 (H)$, then $f$ is weakly l.s.c.
\end{proposition}
