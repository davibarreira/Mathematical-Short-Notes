\chapter{Real Analysis}

\section{Number Systems - Real and Complex}

\subsection{Supremum and Infimum}

\begin{definition}[Ordered Set]
	Let $S$ be a set. An order relation on $S$ is denoted by $<$ with:
	\begin{enumerate}[(i)]
		\item If $x,y  \in S$ then either $x<y, x=y$ or $x>y$;
		\item If $x,y, z \in S$ then $x < y, y < z \implies x < z$.
	\end{enumerate}
	A tuple $(S, <)$ is called an ordered set.
\end{definition}

\begin{definition}[Supremum and Infimum]
	For an ordered set $(S,<)$, we say that $\alpha$ is the supremum of $E \subset S$
	($\alpha = \sup_{x \in S} E$) if $\alpha$
	is the least upper bound of $S$, i.e.
	\begin{enumerate}[(i)]
		\item For any $x \in E$, then $x \leq \alpha$;
		\item If $y < \alpha$ then $y$ is not an upper bound of $E$;
	\end{enumerate}
	The Infimum is the analogous definition, but on the other direction.
\end{definition}

\begin{definition}[Least-upper-bound property]
	An ordered set $(S,<)$ is said to have the Least-upper-bound property if
	for any non-empty and bounded set $E \subset S$, then $\sup E$ exists
	and belongs to $S$.
\end{definition}

\begin{definition}[Ring]
	A ring is a triple $(R, +, \cdot)$ where $R$ is a set with at least two elements,
	and that contains two operations called addition ($+$) and multiplication ($\cdot$),
	where they satisfy the following conditions:

	\textbf{Addtion:}
	\begin{enumerate}[(i)]
		\item If $x, y \in F$, then $x+y \in F$;
		\item $x + y = y + x$;
		\item $(x + y) + z = x + (y + z)$;
		\item There exists a null element $0 \in F$ such that $0 + x = x$ for every $x \in F$;
		\item For every $x\in F$ there exists a $-x \in F$ such that $x + (-x) = 0$.
	\end{enumerate}

	\textbf{Multiplication:}
	\begin{enumerate}[(i)]
		\item If $x, y \in F$, then $x\cdot y \in F$;
		\item $x \cdot y = y \cdot x$;
		\item $(x \cdot y) \cdot z = x \cdot (y \cdot z)$;
		\item There exists an identity element $1\in F$ such that $1 \cdot x = x$ for every $x \in F$;
	\end{enumerate}

	\textbf{A/M}
	\begin{enumerate}[(i)]
		\item Multiplication is distributive in terms of addition, i.e.
		      $x\cdot(y+z) = x \cdot y + x \cdot z$.
	\end{enumerate}
	If the commutative property for multiplication is not satisfied, we have a \textit{non-commutative ring}.
\end{definition}

\begin{definition}[Field]
	A field is a ring $(F, +, \cdot)$ with the extra condition that
	every $x \neq 0 \in F$ has  an inverse element $1/x \in F$ such that $x \cdot (1/x) = 1$.

	If we add an order relation to $F$, then $(F, +, \cdot, <)$ is an ordered field.
\end{definition}

\begin{example}
	Note that $\mathbb Z$ is a ring, since the usual sum and multiplication
	satisfies all the properties. Yet, it's not a field, since the inverse
	elements are not part of $\mathbb Z$. It's easy to show
	that both $\mathbb Q$ and $\mathbb R$ are fields. More surprisingly,
	the space $\mathbb Z_{p}$, known as the modulo of $p$ where $p$ is a prime number, is also
	a field.
\end{example}

\subsection{Sequences and Limits}

We'll construct the Real numbers using Cauchy sequences.

\begin{definition}[Equivalence Relation]
	Let $X$ be a set. The symbol $\sim$ is a set on $X \times X$, where
	for $x,y \in X$, then $x \sim y$ means that $(x,y) \in \sim$. Hence,
	we say that $\sim$ is an \textit{equivalence relation} on $X$, if $\sim$ satisfies
	the following properties:
	\begin{enumerate}
		\item \textit{reflexive:} $\forall x \in X, x \sim x$;
		\item \textit{symmetric:} if $x \sim y$, then $y \sim x$;
		\item \textit{transitive:} if $x \sim y$ and $y \sim z$, then $x \sim z$.
	\end{enumerate}
\end{definition}

\begin{definition}[Equivalence Class]
	Given $x \in X$, the \textit{equivalence class} of $x$ with respect
	to an \textit{equivalence relation} $\sim$ is
	\begin{equation}
		[x]:=\{
		y \in X: y \sim x
		\}.
		\label{eq:eqclass}
	\end{equation}
\end{definition}

\subsection{Topology}

\begin{definition}[Topological Space and Open Sets]
	$(X,\mathcal T)$ is a topological space where $\mathcal T$ is the collection of \textit{open
		sets} in $X$, such that:
	\begin{enumerate}[(i)]
		\item $X \in \mathcal T$;
		\item Se $A_1,...,A_n \in \mathcal T \implies \cap_{i=1}^n A_i \in \mathcal T$;
		\item Se $A_\alpha \in \mathcal T$ for any $\alpha \in \Lambda \implies \cup_{\alpha \in \Lambda} A_\alpha \in \mathcal T$;
	\end{enumerate}
	$\mathcal T$ is the topology of $X$.
\end{definition}

\begin{definition}[Closed Sets]
	Given a topologial space $(X, \mathcal T)$, we say that a set $F \subset X$ is closed
	if $F^c \in \mathcal T$.
\end{definition}

\begin{definition}[Interior and Closure]
	Let $(X, \mathcal T)$. Take a set $A \subset X$. The interior of $A$ is
	the union of all open sets that are subsets of $A$, and it's denoted by
	$A^{\circ}$. The closure of $A$ is the intersection of all closed sets containing
	$A$, denoted by $\bar A$. E.i.
	\begin{equation}
		A^{\circ} := \bigcup_{
			U \subset A , \ U \in \mathcal T
		} U,\quad
		\bar A:= \bigcap_{
			F \supset A , \ F^c \in \mathcal T
		} F
	\end{equation}
\end{definition}

\begin{definition}[Metric]
	Let $X$ be a space. A function $d:X\times X \to \mathbb [0,+\infty)$ is called a metric if
	\begin{enumerate}[(i)]
		\item $d(x,y) = 0 \iff x =y$;
		\item $d(x,y) = d(y,x)$;
		\item $d(x,z) \leq d(x,y) + d(y,z)$.
	\end{enumerate}
	If $d$ satisfies only (i) and (ii) then it's called a pseudo-metric.
\end{definition}

\begin{definition}[Metric Space]
	A metric space is the tuple $(X,d)$, where $d$ is the metric over $X$,
	e.g. $X=\mathbb R$ and $d(x,y) = |x-y|$.
\end{definition}

\begin{definition}[Open Ball]
	Let $(X,d)$ be a metric space, and $x \in X, r >0$. The open ball is
	\begin{equation}
		B_r(x):= \{y \in X: d(y,x) < r\}.
	\end{equation}
\end{definition}

\begin{definition}[Open Family Induced by $d$]
	Let $(X,d)$ be a metric space. $x$ is an interior point of $A\subset X$
	if there exists an open ball $B_r(x)\subset A$.
	The set of interior points of $A$ is denoted by $A^\circ$.
	We can define a family of open sets by defining that
	$A$ is open if $A = A^\circ$.

	Hence, the open family $\mathcal O$ induced by $d$ is the set
	\begin{equation}
		\mathcal O := \{A \in \mathcal O : \exists x\in A, r>0 \text{ such that } B_r(x) \subset A\}.
	\end{equation}

	Note that in this way,
	the notion of being an open set is directly related to the metric $d$,
	since it requires that in every open set there is an open ball inside,
	which was defined using $d$.
	One can check that defining an open family this way satisfies the
	definition \ref{openfamily}.
\end{definition}

\subsection{Construction of Real}


\begin{theorem}[Compactnes + Unique Subsequences]
	Let $x_n \in K \subset \mathbb R^n$, with $K$ compact.
	If $\exists x$ such that if a subsequence
	converges, then it converges to $x$ (i.e.
	all convergent subsequences have a unique
	limit point $x$), then $x_n \to x$.
\end{theorem}
\begin{proof}
	Suppose that $x_n \nrightarrow x$. Then, $\exists \varepsilon > 0$ such
	that for every $N \in \mathbb N$ there exist $n \geq N$ such that
	$d(x_n , x) \geq \varepsilon$. Now, construct a subsequence
	$x_{n_k}$ such that $d(x_{n_k},x)\geq \varepsilon$ for every $x_{n_k}$.
	Since $(x_{n_k}) \subset K$ compact, then there exists a subsequence
	$x_{n_{k_j}}$ that converges, therefore, $x_{n_{k_j}} \to x$, which
	is a contradiction, since $d(x_{n_{k_j}},x) \geq \varepsilon$.
\end{proof}

\subsection{Differentiation}

\begin{theorem}[Mean Value Theorem Inequality]
  Let $\bm f: [a,b] \to \mathbb R^k$ with $\bm f$ differentiable in $(a,b)$.
  Then, there exists $c \in (a,b)$ such that
  \begin{displaymath}
    \vert \bm f(b) - \bm f(a) \vert \leq (b-a)\vert \bm f'(x) \vert.
  \end{displaymath}
  \label{thm:almosttvm}
\end{theorem}
