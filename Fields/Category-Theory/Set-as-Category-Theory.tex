\section{What are Sets?}

This section is based on \citet{leinster2014rethinking}.

When defining \textit{small} and \textit{locally small} categories, we need to
differentiate between a \textit{class} and a \textit{set}. Anyone familiar with
Russel's paradox on the set of all sets can appreciate why such distinction might be
relevant.

One way to solve Russel's paradox was via Zermelo-Frankael and Choice (ZFC) axioms.
Instead of strictly defining a set, the ZFC define what properties a set should have.
Although this approach is the one assumed by most mathematicians, what ZFC
calls a ``set'' does not actually match with how mathematicians use it.
An example of the oddity in the definition of set's by ZFC is that elements of sets are also
sets, so one could ask questions like ``what are the elements of \pi?'' \citep{leinster2014rethinking}.

Hence, instead of ZFC, we'll introduce here William Lawvere axioms as presented in \citet{leinster2014rethinking}. 
Although less common, such system is more in sync with Category Theory, which is the subject at hand, and at the same
time, it seems to more accurately describe what we mean by ''sets``.

\subsubsection{Lawvere's Elementary Theory of the Category of Sets (ETCS)}

As we said, to define a set we'll actually determine the properties that such object
possesses. Thus, anything with such properties we'll be called a set. Of course,
when stating such definition, we'll use terms that are again not tightly defined.
But this is just part of life, since without such artifice, we would end up with
circular definitions.

Let's now introduce the 10 axioms that make ETCS. This system of axioms is actually
weaker (more general) than ZFC, and it can be shown to correspond to
``Zermelo with bounded comprehension and choice'' \citep{leinster2014rethinking}.

Although this axiomatization does not require Category Theory, we'll see that in
some sense it has a categorical ``flavor'' to it.

Before stating the axioms, let's present some definitions that we'll be used in
the axioms themselves. Note that these definitions only make sense once the axioms
are established. But we present them now in order to make the exposition of ETCS
cleaner.

\begin{definition}[Terminal Set]
    A set $T$ is called \textbf{terminal} in ETCS if for every set $X$ there is
    only one function $f:X \to T$.
\end{definition}
The terminal set is a way to define a single element set without relying on the definition
of an element. In order to prove that this is indeed the case,
we would need to clarify when two functions are the same, which will only be done
after we present our axioms. It can be shown that every terminal set is
unique up to an isomorphism, so one could use $T$ to represent every terminal set.

Interestingly, if we are working in a context with a restricted
collection of functions, then, a set $T$ may behave as a single element set, while
it may have multiple elements in another context.
Consider for example, that $T = [0.5,1]$, and we are in the context of functions
that return natural numbers. Thus, for any set $X$, there exists only one function
$f:X \to T$, which always returns $1$.

As we've seen, the category of sets ($\bm{Set}$) will consist of $\langle Ob_{\bm{Set}}, Mor_{\bm{Set}}\rangle$,
where $Ob_{\bm{Set}}$ is the collection of every set, and $Mor_{\bm{Set}}$ is the collection of every function.
In the ETCS, the collection of every set will not be a set itself.

\begin{definition}[Element of a Set]
    Given a set $X$, we write $x \in X$ to mean $x:T \to X$ where $T$
    is a terminal set.
\end{definition}
Note that in this definition of an element, what we call an element of $X$
is actually a function. Also, for $f: X \to Y$,
then $f \circ x$ is a function from $T$ to $Y$, i.e. it is an element of $Y$,
which we write as $f(x) \in Y$.

\begin{definition}[Cartesian Product]
    Given sets $X$ and $Y$. The Cartesian product of $X$ and $Y$
    is a set $P$, with functions
    $p_1 : P \to X$ and $p_2 : P \to Y$, such that for any set $Z$
    and functions $f_1 : Z \to P$ and $f_2 : Z \to P$, there exists
    a unique function $F = (f_1,f_2) : Z \to P$ where
    \begin{displaymath}
        p_1 \circ (f_1,f_2) = f_1, \quad p_2 \circ (f_1, f_2) = f_2.
    \end{displaymath}
\end{definition}
Note that the Cartesian Product determines not only a product set, but also
the projection functions. Similar to terminal sets,
for any sets $X$ and $Y$, the triple $(P,p_1,p_2)$
are unique up to an isomorphism. Thus, we could fix $(P,p_1,p_2)$ to be
represented by $(X\times Y, \pi_1^{X\times Y}, \pi_2^{X\times Y})$.

\begin{definition}[Function set]
    Let $X$ and $Y$ be two sets. A \textbf{function set} from
    $X$ to $Y$ is a tuple $(F, \varepsilon)$, where $F$ is a set and $\varepsilon$
    is a function $\varepsilon: F \times X \to Y$ such that
    for all sets $Z$ and functions $q:Z \times X \to Y$, 
    there exists a unique function $\overline q: Z \to F$ with
    $q(t,x) = \varepsilon(\overline q(t), x)$ for all $t \in Z$
    and $x \in X$.
\end{definition}

\begin{definition}[Inverse Image]
    Let $f: X \to Y$ be a function and $y \in Y$. The \textbf{inverse image}
    of $y$ under $f$ is a tuple $(A, j)$ where $A$ is a set and $j:A \to X$
    is a function such that $f \circ j (a) = y$ for every $a \in A$. Also,
    for every set $Z$ and function $q : Z \to X$ such that 
    $f(q(t)) = y$ for every $t \in Z$, there is a unique function
    $\overline q : Z \to A$ such that $q = j \circ \overline q$.
\end{definition}
Again it can be shown that inverse images are unique up to an isomorphism.

\begin{definition}[Injection]
    An injection $j:A \to X$ is a function with the property that
    $j(a) = j(a') \implies a = a'$ for every $a,a' \in A$.
\end{definition}

\begin{definition}[Surjection]
    A surjection $s:X \to Y$ is a function such that for every
    $y \in Y$ there exists an $x \in X$ such that $s(x)=y$.
\end{definition}

\begin{definition}[Right inverse]
    The right inverse of a function $s:X \to Y$ is a function
    $i : Y \to X$ such that $s \circ i = 1_Y$.
\end{definition}

\begin{definition}[Subset Classifier]
    The tuple $(\bm{2}, t)$ where $\bm 2$ is a set and $t \in \bm 2$ is called
    a subset classifier if for all sets $A, X$ and injections
    $j : A \to X$, there is a unique function $\chi : X \to \bm 2$,
    such that $(A,j)$ is an inverse image of $t$ under $\chi$.
\end{definition}
Note that in the definition above, the function $\chi$ can be seen as a characteristic
function. Suppose that we wish to define $\chi_A$. Hence, it's required that there exists
a set $\bm 2$ with $t \in \bm 2$ such that $\chi_A(j(a)) = t$ for every $a \in A$.

\begin{definition}[Natural Number System]
    A natural number system is a triple $(N, 0, s)$ where $N$
    is a set, $0 \in N$ and $s: N\to N$, such that
    for any set $X$, $a \in X$ and $r : X \to X$, there is a unique
    function $x:N \to X$ where $x(0) = a$ and 
    $x(s(n))= r(x(n))$ for every $n \in N$.
\end{definition}
This comes from the idea that $s(n) \cong n+1$, that $x(0) \cong x_0$ and
$x_n \cong x(s(n-1)) \cong r(x_n) \cong r(x(n))$. Once more, natural
number systems are unique up to an isomorphism.

After all this definitions, we can finally state the axioms for Set Theory.
\begin{definition}[ETCS]
     Lawvere's Elementary Theory of the Category of Sets consists on the following axioms:
    \begin{enumerate}[(i)]
        \item For all sets $W,X,Y,Z$, and functions
            $f:W\to X$, $g:X\to Y$, $h:Y \to Z$, we have
            \begin{displaymath}
                h \circ (g \circ f) = (h\circ g) \circ f.
            \end{displaymath}
        For every set $X$ and $Y$ and function $f: X \to Y$, there exist the identity functions
        $1_X$ and $1_Y$, such that
            \begin{displaymath}
                f \circ 1_X = f = 1_Y \circ f.
            \end{displaymath}

        \item There exists a terminal set $T$.

        \item There exists a set with no elements, i.e. an empty set denoted by $\varnothing$.
        
        \item For sets $X,Y$ and functions $f:X \to Y$ and $g:X\to Y$, if
            $f(x) = g(x) \forall x \in X$, then $f = g$.

        \item Every pair of sets has a Cartesian product.

        \item For all sets $X$ and $Y$, there is a function set from $X$ to $Y$.

        \item For every $f : X \to Y$ and $y \in Y$, there is an inverse image
            of $y$ with respect to $f$.

        \item There exists a subset classifier. This can be thought as saying that
            for every set we can construct a characteristic function.

        \item There exists a natural number system.

        \item Every surjection has a right inverse.
    \end{enumerate}
\end{definition}

As we pointed out, these axioms are actually weaker than ZFC, but with one extra axiom,
it can be shown to be as strong as ZFC. The last axiom is the one related to the Axiom of Choice.
The first axiom states that sets form a category, and the following axioms distinguish this
category from others.

With these axioms stated, we can now define the notion of a subset, and clearly
differentiate objects that are and that aren't actually sets.
One might think that ``anything we can reasonably conceive'' must be a set.
But this is not the case.
\begin{definition}[Subset]
    Given a set $X$, a subset of $X$ is a function
    $f:X \to \bm 2$. The subset $\chi_A : X \to \bm 2$ is written
    as $A \subset X$, where $\chi_A$ is the characteristic function
    with $\chi_A^{-1}(t) = A$.
\end{definition}

\begin{corollary}
    A set $T$ is terminal if and only if it has only a single element.
\end{corollary}
\begin{proof}
    $\implies)$ If $T$ is terminal, then for any set $X$, we have a unique $f:X\to T$.
    For $t_1,t_2 \in T$, then $t_1 : T' \to T$ and $t_2: T' \to T$ where $T'$ is a terminal set.
    Note that $f:T' \to T$ is unique, hence, $t_1 = t_2$, meaning that $T$ has only a single element.

    $\impliedby)$ If $T$ has a single element $t \in T$, then for a set $X$, take $f_1 : X \to T$ and
    $f_2 : X \to T$. Since $T$ has only one element, then $f_1(x) = t = f_2(x)$, which, by Axiom 3,
    implies that $f_1 = f_2$.
\end{proof}

