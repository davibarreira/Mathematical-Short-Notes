\newpage
\section{Programming with Category Theory}

This section is mainly based on \citet{milewski2018category},
Programming With Categories by Brendan Fong et. al.

Here, we'll try to give examples both on Haskell
(the original language for \citet{milewski2018category})
and Julia.

\subsection{Comments on Yoneda Lemma}

Note that a every element in a set $X$ is isomorphic to
a morphism (i.e. function) $f: 1 \to X$
\footnote{Remember, the set $1$ is the set
containing one object and the identity morphism.}
This means that a set $X$ is entirely determined by
the morphisms from object $1$, which is the initial
object of category $\mathbf{Set}$.


\textbf{CHECAR INFO ABAIXO}.
The fact that the $1$ object is enough to characterize the
objects of $\mathbf{Set}$ is not true for every other category.
Yet, for every other locally small category
(i.e. every class of morphisms from one object to another is a set),
an object is completely defined by the set of all morphisms arriving at it,
in other words, $Hom(-, A) \cong A$.

More than that, the Yoneda Lemma tells us that for
two objects $A$ and $B$, and a morphism $f:A\to B$, there
is a way 

\subsection{Setting Up Haskell}



