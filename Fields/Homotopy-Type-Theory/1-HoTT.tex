\section*{Notation}

Perhaps to my own detriment, I'll deviate from the common
notation from HoTT and use something more in line with
programming. More specifically, Julia programming.

\section{Introduction}
Type theory has many relations to the foundations of mathematics,
and hence, to logic.

First, in type theory, every \textit{type} corresponds
to a \textit{proposition}. The basic \textit{judgment}
of type theory is written as $a::A$, and it represents
``$A$ has a proof''. The $a$ in $a::A$ is called a \textit{witness}.

But what are propositions and judgements? Think of a proposition
as a statement which can be proven, disproven, assumed, and so on.
A judgement is a statement about a proposition, e.g.
``proposition $A$ has a proof''.

Another important aspect to note is the notion of equalities.
The proposition of an equality is similar to the $==$
in programming, meaning, we are ``testing'' whether two
variables are the same. While the judgement is akin to
assigning (defining) an equality:

\hspace{100mm}
\begin{lstlisting}[language=JuliaLocal, style=julia]
x = 10 # judgment
y = 10 # judgment
x == y # proposition
\end{lstlisting}

Since $x==y$ is a proposition, it also means that, in type theory,
this is a type. Thus, we can have $a:(x==y)$.
