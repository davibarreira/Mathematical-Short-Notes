Notes for the 2021 PhD course of Ordinary Differential Equations
at FGV/EMAp. The main reference used was \citet{doering2008equaccoes}.

\section{Introduction}

\subsection{Initial Definitions}
\begin{definition}[ODE]
  An Ordinary Differential Equation is an expression
  \begin{displaymath}
    \frac{d^k x(t)}{dt^k} = x^{(k)}(t) = F(t,x,x',...,x^{(k-1)}),
  \end{displaymath}
  where $k,d \in \mathbb N, t \in \mathbb R, x \in \mathbb R^d$ and $F:U\to \mathbb R^d$ is a continuous
  function defined on an open set $U \subset \mathbb R^{1+d\cdot k}$.
  We call $k$ the \textit{order} and $d$ the \textit{dimension}.
\end{definition}

If we take the canonical coordinates of, say, $\mathbb R^n$, then
\begin{align*}
  x(t) = (x_1(t), x_2(t),...,x_n(t)),\quad \text{and }
  f(t,x) = (f_1(t,x), f_2(t,x),...,f_n(t,x))
\end{align*}
can instead be written as a system of differential equations
\begin{align}
  \begin{cases}
    x'_1(t) = f_1(t,x_1(t),...,x_n(t)),\\
    x'_2(t) = f_2(t,x_1(t),...,x_n(t)),\\
    \ \vdots\\
    x'_n(t) = f_n(t,x_1(t),...,x_n(t)).\\
  \end{cases}
\end{align}


\begin{definition}[Solution to ODE]
  A solution to an ODE $F:U\to \mathbb R^{1+d\cdot k}$ is a function
  $\gamma : I \to \mathbb R^d$ such that
  \begin{enumerate}
    \item $\gamma \in C^k$;
    \item $I\subset R$ is an open interval such that $\forall t \in I$,
      $(t,\gamma(t),\gamma'(t),...,\gamma^{(k-1)}(t))\subset U$;
    \item $\frac{d^k\gamma(t)}{dt^ k} = \gamma^{(k)}(t) = F(t,\gamma'(t),...,\gamma^{(k-1)}(t)), \ \forall t \in I$.
  \end{enumerate}
\end{definition}

\subsection{Classification}
There are many ways in which we can ``classify'' an ODE.
We say that an ODE is \textit{normal} if we can explicitly write $x'$, 
i.e. $x' = f(t,x)$. In these notes, we are mainly interested in this type
of ODE which are easier to work with.

If $x' = f(t,x) = f(x)$, then we say that our equation is \textit{autonomous},
meaning that is does not depend on time. Although, it can be shown that
any non-autonomous ODE can be written as an autonomous ODE. Consider that
$x'(t) = f(x,t)$ and make $X = (t,x)$ with $F(X) = (1, f(t,x))$. Thus,
\begin{displaymath}
  X'(u)  = F(X) = (1, f(t(u), x(u))).
\end{displaymath}
Note that if $X(u) = (t(u),x(u))$ is a solution to the above ODE, then
for $t(t_0) =t_0$, we have that $t'(u) = 1 \implies t(u)= u$, hence
$x'(u) = f(u,x(u))$.

A similar thing can be said about the order of an ODE.
Every second order (or more) ODE can be transformed into a first order problem.
Let
\begin{align*}
  \begin{cases}
 y'' = g(t,y,y'), \\
 y(t_0) = y_0, \\
 y'(t_0) = y_1.
  \end{cases},
\end{align*}
such that $g:U\subset \mathbb R^3 \to \mathbb R$. Then, define $x_1(t)=y(t)$
and $x_2(t) = y'(t)$. Hence, we now have a system of first order differential
equations
\begin{align*}
  \begin{cases}
 x_1' = x_2, \\
 x_2' = g(t,y,y'), \\
 x_1(t_0) = y_0, \\
 x_2(t_0) = y_1.
  \end{cases}
\end{align*}

Note that we can do the same procedure if the ODE has a larger order. Thus,
every ODE can be represented by a first order ODE.

Hence, we can restrict our study to the case of first order autonomous equations.

\section{Existence and Uniqueness}

Let's begin by proving that indeed there exists solutions to Ordinary
Differential Equations, and that, under stronger assumptions, we
can even show that such solutions are unique.

\begin{shaded}
  \textbf{Remember} that we say that a function
  $f:X\to Y$ between two metric spaces is \textit{Lipschitz} if there exists
  $C > 0$ such that
  \begin{displaymath}
    d_Y (f(x_1), f(x_2)) \leq C d_X(x_1,x_2) ,\quad \forall x_1, x_2 \in X,
  \end{displaymath}
  where $C$ is called the Lipschitz constant. Now, an analogous definition
  are the \textit{locally Lipschitz} functions. We say that the function
  is locally Lipschitz if for every $x_0 \in X$, there exists an open set $U \in X$
  and a constant $C_{x_0}>0$ such that
  \begin{displaymath}
    d_Y (f(x_1), f(x_2)) \leq C_{x_0} d_X(x_1,x_2) ,\quad \forall x_1, x_2 \in U.
  \end{displaymath}

  Let $F:\mathbb R^k \to \mathbb R^n \in C^1$. By the Mean Value Inequality,
  we have that
  \begin{displaymath}
    \Vert F(t,x_1)   - F(t,x_2) \Vert \leq \Vert x_1 - x_2 \Vert M.
  \end{displaymath}
  

\end{shaded}

\subsection{Picard's Theorem}

Consider that we wish to solve the initial value problem
\begin{displaymath}
 x'(t)  = f(t,x), \quad x(t_0) = x_0,
\end{displaymath}
where $x: \mathbb R \to \mathbb R^n$, thus, $f:\mathbb R^{n+1} \to \mathbb R^n$.
Since $x$ is differentiable, then $x(t)$ is a continuous path, hence,
in the compact set $I = (t_0, T)$, the function is uniformly continuous and
we can apply the Fundamental Theorem of Calculus to say that
\begin{displaymath}
 L_x(t) := x_0 + \int^t_{t_0} x'(s) ds = x(t).
\end{displaymath}


