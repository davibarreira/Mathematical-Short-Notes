\section{Introduction}

Survival Analysis is about studying the time until
an event occurs. Usually, one will be interested in
predicting such time, for example, how long will
it take for case to be given a sentence in the
judiciary system.

An important concept in Survival Analysis is the idea
of Censoring. This refers to data in which the information
is unknown for some reason. For example, suppose that
for some reason a case is dropped by the judiciary before
the sentence. Thus, the output for such case was ``censored''
since we can't observe when it would've occur.

In this scenario, there are three types of censoring,
the right censoring, which we described in the example,
where the occurrence of the event is unseen.
The left censoring occurs when the starting time is unknown.
Consider for example that we know that a case entered the
justice system in January, but we don't know which day.
Lastly, interval censoring is, for example,
the case in which we can only know the monthly information,
so we know the case entered in January, and it was sentenced
in December, but we don't know the day.

Let's denote $T_i$ the random variable for the survival time
of event $i$, and $C_i$ the right censoring time.
Thus, we defined another random variable $\Delta_i$
such that $\Delta_i(\omega) = 1$ if $T_i(\omega) \geq C_i(\omega)$
and 0 otherwise.

The survival function of $T$ is $S(t) = P(T>t) =1 - P(T\leq t)$.

\begin{definition}[Hazard]
	Given a r.v. $T$ and a survival function $S$,
	suppose that $T$ is continuous with pdf $p(t)$.
	Thus, $h(t):=p(t \mid T \geq t)$. Note that this
	is similar, but not equal, to the conditional
	distribution $p(t \mid T \geq t_0)$, since our
	$t_0$ is also varying.

	The hazard can also be expressed as:
	\begin{displaymath}
		h(t) = \frac{p(t)}{S(t)}.
	\end{displaymath}

\end{definition}

The hazard is not a probability distribution, but,
in a sense, it's the rate that at a time $t$ one will
fail. At first, one might think that such rate would
be $p(t)$ itself, but note, for example, that if
$T \sim \text{Exp(\lamba)}$, the peak is right in the
beginning, but the chance of actually failing in the
beginning is zero. Thus, $p(t)$ is not the rate of
failure experience at such time, but it's more like
the rate at which your risk of failure grows.

Hence, in an exponential distribution the risk of failure
grows faster in earlier times, but the actual rate of
failure is given by the hazard function.


