\section{Introduction}

This chapter draws heavily from \citet{munkres2018elements}.

\section{Functions, Order and Equivalent Classes}

\subsection{Functions}

\begin{definition}[Functions]
  A function $f$ from a set $X$ to $Y$ is an assignment of an element of $Y$ to each element of $X$,
  denoted by $f: X \to Y$.
  The set $X$ is called domain, and $Y$ the codomain. The set $f(X):=\{f(x) :  x \in X\}$ is the
  image of $f$.

  A partial function from $X$ to $Y$ is a function from a subset $S \subset X$, i.e.
  a partial function from $X$ is not defined for every element of $X$.
  This type of object appears very often in Computation Theory, where partial functions
  usually take the role of a program, and when the program halts, this can be thought
  as the partial function not being defined for that specific entry of the domain.
  In this area, it's common to make the distinction of partial functions and complete functions,
  where the notion of a complete function is what we are calling a function.
\end{definition}

\begin{definition}[Injective, Surjective, Bijective]
  A function $f:X \to Y$ is:
  \begin{enumerate}[(i)]
    \item Injective  if for any $x_1,x_2 \in X$ with $x_1 \neq x_2$ then $f(x_1) \neq f(x_2)$;
    \item Surjective (onto) if for every $y \in Y$ there exists $x \in X$ such that $f(x)= y$;
    \item Bijective (one-to-one) if $f$ is both injective and surjective.
  \end{enumerate}
\end{definition}

\begin{definition}[Inverse Image]
  Let $f$ be a function from $X$ to $Y$. For every subset $B \subset Y$, the
  inverse image (preimage) of $B$ with respect to $f$ is
  \begin{equation}
    f^{-1}(B) := \{x \in X : f(x) \in B\}.
    \label{eq:invimage}
  \end{equation}
\end{definition}

The inverse image has some nice properties that will be useful in many subsequent proofs.
\begin{proposition}[Properties of Inverse Images]
  Let $f: X \to Y$ and $\{U_\lambda\}_{\lambda \in \Lambda}$ be a family of subsets of $Y$.
  We have the following identities:
  \begin{enumerate}[(i)]
    \item Union and Intersection:
      \begin{displaymath}
        f^{-1}\left(\bigcup_{\lambda \in \Lambda} U_\lambda\right) =
        \bigcup_{\lambda \in \Lambda} f^{-1}(U_\lambda), \quad
        f^{-1}\left(\bigcap_{\lambda \in \Lambda} U_\lambda\right) =
        \bigcap_{\lambda \in \Lambda} f^{-1}(U_\lambda).
      \end{displaymath}
    \item Complement:
      \begin{displaymath}
        (f^{-1}(U))^c =
        f^{-1}(U^c).
      \end{displaymath}
    \item Set difference and symmetric difference:
      \begin{align*}
        f^{-1}(U \setminus V) &=
        f^{-1}(U) \setminus f^{-1}(V), \\
        f^{-1}(U \Delta V) &=
        f^{-1}(U) \Delta f^{-1}(V).
      \end{align*}
  \end{enumerate}
\end{proposition}
\begin{proof}
  Just use the definition.
\end{proof}

\subsection{Order and Equivalence Relations}

\begin{definition}[Relation]
  A relation on a set $X$ is a set $R \subset A \times A$.
\end{definition}

This notion of a relation is very common in mathematics, and appears
implicitly in many concepts.

\begin{definition}[Total Order Relation]
	A total order relation on a set $X$ is denoted by $<$ with:
	\begin{enumerate}[(i)]
		\item If $x,y  \in X$ then either $x<y, x=y$ or $x>y$;
		\item $x < y, y < z \implies x < z$.
	\end{enumerate}
	A tuple $(X, <)$ is called an ordered set.
\end{definition}

\begin{definition}[Partial Order Relation]
  An order relation $\prec$ on a set $X$ is called a strict partial order if:
	\begin{enumerate}[(i)]
		\item The relation $x \prec x$ is never true;
		\item If $x \prec y, y \prec z$ then $x \prec z$.
	\end{enumerate}
	A tuple $(X, <)$ is called an ordered set.
\end{definition}
Note that in the partial order relation, not every element of $X$
must be comparable.

\begin{definition}[Well-Ordered Set]
  A set $X$ with a total order relation $<$ is said to be well-ordered
  if every nonempty subset of $A$ has a smallest element.
\end{definition}

Note that $\mathbb N$ is well-ordered. $\mathbb N \times \mathbb N$ can
also be well-ordered if we consider the dictionary order.
\begin{definition}[Dictionary Order Relation]
  Let $(A, <_A)$ and $(B, <_B)$ be two ordered sets. For $A \times B$,
  the dictionary order relation is $<_D$ where for $x=(x_1,x_2) \in A\times B$ and
  $y = (y_1,y_2) \in A\times B$,
  we say that $x <_D y$ if $x_1 <_A x_2$, or, if $x_1 = x_2$, then $y_1 <_B y_2$.
  In words, the dictionary order sorts first by the first order relation and then
  by the second.
\end{definition}

\begin{definition}[Equivalence Relation]
	Let $X$ be a set. The symbol $\sim$ is a set on $X \times X$, where
	for $x,y \in X$, then $x \sim y$ means that $(x,y) \in \sim$. Hence,
	we say that $\sim$ is an \textit{equivalence relation} on $X$, if $\sim$ satisfies
	the following properties:
	\begin{enumerate}
		\item \textit{reflexive:} $\forall x \in X, x \sim x$;
		\item \textit{symmetric:} if $x \sim y$, then $y \sim x$;
		\item \textit{transitive:} if $x \sim y$ and $y \sim z$, then $x \sim z$.
	\end{enumerate}
\end{definition}

\begin{definition}[Equivalence Class]
	Given $x \in X$, the \textit{equivalence class} of $x$ with respect
	to an \textit{equivalence relation} $\sim$ is
	\begin{equation}
		[x]:=\{
		y \in X: y \sim x
		\}.
		\label{eq:eqclass}
	\end{equation}
\end{definition}

One very useful property of equivalence classes is that they partition the set in
which they are defined. For example, consider an equivalent relation $\sim$ on $X$.
Every element $x \in X$ belongs to the equivalence class $[x]_{\sim}$, and each
class is either the same of disjoint. We can easily show this, just note that
\footnote{In the proof below there is a small technical issue where the first
\textit{iff} should actually be $\exists y \in [x]: y \notin [x']\vee \exists y'
\in [x']: \notin [x]$. But we omitted this ``or'' condition for brevity, and used
instead the ``w.g.l'' which stands for ``without loss of generality''.}
\begin{align*}
  [x] \neq [x'] \iff \small{\text{w.l.g. }} \exists y \in [x], y \notin [x'] &\iff
y \not \sim y', \forall y' \in [x'] \\
&\iff [x] \cap [x'] = \varnothing.
\end{align*}

\subsection{A bit of Algebra}

\begin{definition}[Ring]
	A ring is a triple $(R, +, \cdot)$ where $R$ is a set with at least two elements,
	and that contains two operations called addition ($+$) and multiplication ($\cdot$),
	where they satisfy the following conditions:

	\textbf{Addition:}
	\begin{enumerate}[(i)]
		\item If $x, y \in F$, then $x+y \in F$;
		\item $x + y = y + x$;
		\item $(x + y) + z = x + (y + z)$;
		\item There exists a null element $0 \in F$ such that $0 + x = x$ for every $x \in F$;
		\item For every $x\in F$ there exists a $-x \in F$ such that $x + (-x) = 0$.
	\end{enumerate}

	\textbf{Multiplication:}
	\begin{enumerate}[(i)]
		\item If $x, y \in F$, then $x\cdot y \in F$;
		\item $x \cdot y = y \cdot x$;
		\item $(x \cdot y) \cdot z = x \cdot (y \cdot z)$;
		\item There exists an identity element $1\in F$ such that $1 \cdot x = x$ for every $x \in F$;
	\end{enumerate}

	\textbf{A/M}
	\begin{enumerate}[(i)]
		\item Multiplication is distributive in terms of addition, i.e.
		      $x\cdot(y+z) = x \cdot y + x \cdot z$.
	\end{enumerate}
	If the commutative property for multiplication is not satisfied, we have a \textit{non-commutative ring}.
\end{definition}

\begin{definition}[Field]
	A field is a ring $(F, +, \cdot)$ with the extra condition that
	every $x \neq 0 \in F$ has  an inverse element $1/x \in F$ such that $x \cdot (1/x) = 1$.

	If we add an order relation to $F$, then $(F, +, \cdot, <)$ is an ordered field.
\end{definition}

\begin{example}
	Note that $\mathbb Z$ is a ring, since the usual sum and multiplication
	satisfies all the properties. Yet, it's not a field, since the inverse
	elements are not part of $\mathbb Z$. It's easy to show
	that both $\mathbb Q$ and $\mathbb R$ are fields. More surprisingly,
	the space $\mathbb Z_{p}$, known as the modulo of $p$ where $p$ is a prime number, is also
	a field.
\end{example}

\subsection{Axiom of Choice and Zorn's Lemma}

From \citet{munkres2018elements}.
Consider the following. We define a set $A$ to be infinite if it is not finite, where
a set $A$ is finite if there exists a bijection $f:A \to \{1,...,n\}$ for an $n \in \mathbb N$.
Hence, an infinite set is one where there is no such bijection.
Now, we wish to prove that there exists a function $f:\mathbb N \to A$ such that
$f$ as an injective function. The question is then, how do we construct a candidate?

A simple procedure is to consider the following. We pick a $a_1 \in A$ and make
$f(1) = a_1$. For $f(2)$, we pick an element from $A \setminus \{1\}$. For $f(3)$
we pick from $A \setminus \{1,2\}$. And so on. It's clear that such function
would be injective and defined for every element of $\mathbb N$. But we can
feel that something is off. We are not explicitly constructing our function.
What is going on here?

This strange procedure of affirming that we could construct such function without
ever specifying exactly how we select the elements is only possible if we
consider the ``controversial'' \textbf{Axiom of Choice}.

\begin{definition}[Axiom of Choice]
  Version from \citet{munkres2018elements}. Given as collection $\mathcal A$
  of disjoint nonempty sets, there exists a set $C$ such that $C$
  has one element of each set of $\mathcal A$, i.e. for $A \in \mathcal A$,
  $C \cap A$ has only one element.
\end{definition}

The use of the Axiom of Choice leads to the proof of the so called
Well-ordering theorem, proved by Zermelo in 1904.
\begin{theorem}[Well-ordering theorem]
  Every set $A$ has an order relation such that $A$ is well-ordered.
\end{theorem}


