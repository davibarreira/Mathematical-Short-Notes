\section{Basics on Point-Set Topology}

Let's now give a brief overview on some general concepts related
to point-set topology.

\begin{definition}[Topological Space]
	$(X,\tau)$ is a topological space where $\tau$ is the collection of open
	sets in $X$, such that:
	\begin{enumerate}[(i)]
		\item $X \in \tau$;
		\item Se $A_1,...,A_n \in \tau \implies \cap_{i=1}^n A_i \in \tau$;
		\item Se $A_\alpha \in \tau$ for any $\alpha \in \Lambda \implies \cup_{\alpha \in \Lambda} A_\alpha \in \tau$;
	\end{enumerate}
	$\tau$ is the topology of $X$.
\end{definition}

\begin{definition}[Hausdorff Space]
    A topological space $(X, \tau)$ is said to be Hausdorf if
\end{definition}


\section{Metric Spaces}

\begin{definition}[Metric]
	Let $X$ be a space. A function $d:X\times X \to \mathbb [0,+\infty)$ is called a metric if
	\begin{enumerate}[(i)]
		\item $d(x,y) = 0 \iff x =y$;
		\item $d(x,y) = d(y,x)$;
		\item $d(x,z) \leq d(x,y) + d(y,z)$.
	\end{enumerate}
	If $d$ satisfies only (i) and (ii) then it's called a pseudo-metric.
\end{definition}

\begin{definition}[Metric Space]
	A metric space is the tuple $(X,d)$, where $d$ is the metric over $X$,
	e.g. $X=\mathbb R$ and $d(x,y) = |x-y|$.
\end{definition}
